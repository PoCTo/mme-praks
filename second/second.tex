\documentclass[10pt, a4paper]{article}
\usepackage{triada}

\graphicspath{{eps/}{png/}}

\renewcommand{\ell}{\mathcal{E}}

\usepackage{delim}
\usepackage{float}
\usepackage{subfigure}

                           
\begin{document}
\thispagestyle{empty}

\begin{center}
\ \vspace{-3cm}

\includegraphics[width=0.5\textwidth]{msu.eps}\\
{\scshape Московский государственный университет имени М.~В.~Ломоносова}\\
Факультет вычислительной математики и кибернетики\\
Кафедра системного анализа



%\vspace{5cm}
\vfill
{\LARGE Лабораторная работа}

\vspace{1cm}

{\huge\bfseries <<Оценка влияния государственной энергетической политики на потенциал экономического роста России>>}
\end{center}

\vspace{2cm}

\begin{flushright}
  \large
  \textit{Студент 415 группы}\\
  Д.~И.~Степенский

  \vspace{5mm}

  \textit{Руководитель практикума}\\
  к.\,ф.-м.\,н., асс. А.~В.~Рудева
\end{flushright}

\vfill

\begin{center}
Москва, 2012
\end{center}

\newpage

\tableofcontents

\newpage

\section{Постановка задачи}
Требуется построить и проанализировать зависимости от "подставить" следующих макроэкономических показателей:
\begin{itemize}
\item темп роста цен;
\item темп роста производства;
\item параметр неэффективности;
\item доля налогов в добавленной стоимости электроэнергетики;
\item доля зарплаты и распределяемой прибыли в добавленной стоимости НГК;
\item доля потребления населения к ВВП;
\item доля государственных расходов в ВВП;
\item доля добавленной стоимости 1-го сектора в ВВП;
\item доля добавленной стоимости 2-го сектора в ВВП;
\item доля добавленной стоимости 3-го сектора в ВВП;
\item отношение инвестиций во 2-ой сектор к выпуску 2-го сектора;
\item отношение инвестиций в 3-ий сектор к выпуску 3-го сектора;
\end{itemize}
а также объяснить влияние параметра неэффективности на макроэкономические модели.

\end{document}